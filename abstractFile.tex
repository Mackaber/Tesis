En el siguiente documento es presentada una propuesta de tesis para la obtención del grado de Maestría en Ciencias Computacionales y trata de resolver un problema de agrupar distintos usuarios para la práctica de idiomas extranjeros por medio de su nivel de experiencia y sus temas de conversación de interés, entre otros factores. Estamos considerando la existencia de “cuartos” donde los usuarios practican el idioma en cuestión, pero habiendo distintas maneras posibles de agrupar usuarios en un momento dado, también hay unas formas mejores que otras, de acuerdo con criterios predefinidos. Para esto se abordan dos posibles escenarios, en el primero se aborda el problema como uno de optimización mono-objetivo u de optimización general donde se pretende maximizar los criterios antes mencionados tomándose como pesos de una misma función. El segundo escenario es abordar el problema como uno de optimización multiobjetivo donde se realizarán pruebas con distintos algoritmos genéticos de los ya existentes para resolver este problema. Finalmente se harán distintas comparaciones entre la implementación del problema mono-objetivo con respecto a los resultados obtenidos de las pruebas usando algoritmos multi-objetivo a través de una herramienta conocida como jMetal que implementa dentro de sí algunos de los algoritmos genéticos más comunes, además de algunas de las herramientas necesarias para medir el rendimiento tanto de los algoritmos para resolver problemas mono-objetivo como para resolver problemas multiobjetivo. Teniendo en mente que el resultado probablemente tienda a ser mejor con una solución multiobjetivo, también se determinará cual de estos algoritmos resulta tener un mejor desempeño para este problema en específico.