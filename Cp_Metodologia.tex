\chapter{Metodología}
\label{chapter:chapter03}

\section{Funciones Objetivo}

En la siguiente sección se describen distintas funciones objetivo para predecir la preferencia del usuario y la estabilidad del grupo, buscando minimizar el valor de cada una de ellas.

\subsection{Tamaño del Grupo}

Para propósitos de la investigación el tamaño ideal del grupo se considera como de \textbf{4 a 5} personas, ya que este tamaño usualmente permite una participación de todos los usuarios que pertenezcan al grupo. Es posible que puedan existir grupos de \textbf{3 o 6} personas, pero esto no es lo ideal, sin embargo se deja abierta la posibilidad en caso de que alguna otra característica como los intereses en común o el estilo de participación resulten más relevantes para el grupo que su tamaño.

La función para calcular este objetivo mide la cercanía al valor \textbf{4.5}, ya que 4 y 5 se consideran igual de ideales y 4.5 es simplemente el promedio entre ambos valores. Esta función esta definida a continuación:\ref{eq_group_size}

\begin{equation} \label{eq_group_size}
    f = | groupSize - 4.5|
\end{equation}

\subsection{Nivel de experiencia}

El nivel de experiencia en el lenguaje hace referencia al estándar CEFR\cite{}. Los valores que se consideran para ello son A1, A2, B1, B2, C1 y C2, por lo que se les asigna un valor numérico a cada uno siendo A1 0 y C2 5. Lo que se busca principalmente es que los usuarios dentro del grupo no tengan mucha variación en cuanto a su nivel de experiencia. De esta forma el vocabulario usado no se considera tan avanzado para integrantes con un nivel bajo ni tan sencillo para integrantes más avanzados. La fórmula\ref{eq_nivel} consiste en una desviación estándar del valor del nivel de cada uno de los integrantes, de forma ideal que sea alcanzado el 0.

\begin{equation} \label{eq_nivel}
    \sigma = \sqrt{\frac{1}{n-1} \sum_{i=1}^n (x_i - \overline{x})^2}
\end{equation}

\subsection{Intereses}

Para calcular la similitud entre intereses se usa una estrategia similar a Madylova y Gunduz \cite{taxonomy_semantic_similarity} donde los intereses son considerados como un vector valores en los que se señala una cantidad que indica el orden jerárquico de acuerdo a la ontología definida por la red social de Facebook, por ejemplo si un usuario tiene como intereses: Los perros, Correr y las Citas, su vector de intereses se estaría dado por los valores en la Figura \ref{fig:interests_table}. En la parte de arriba se indica el valor de la jerarquía, en la parte de abajo se normaliza usando \(1/(1+v)\), después se eligen los valores mayores de cada una de las columnas de esta forma da el vector resultante como se muestra en la Figura \ref{fig:interests_vector} \\

% Cambiar por tabla
\begin{figure}
    \centering
    \includegraphics{interests_table.png}
    \caption{Tabla de valores de Intereses}
    \label{fig:interests_table}
\end{figure}

% Cambiar por tabla
\begin{figure}
    \centering
    \includegraphics{interests_vector.png}
    \caption{Vector de Intereses}
    \label{fig:interests_vector}
\end{figure}

Para calcular que tan similar es un usuario con respecto a otro se usa la formula de distancia cosenoidal, que se muestra en la ecuación \ref{eq_cos_sym} donde \(A\) es el vector de intereses del primer usuario y \(B\) es el vector de intereses del segundo. Cabe destacar que es muy frecuente que los vectores de intereses de los usuarios contengan ceros al momento de compararlos, de tal manera que si el resultado de la función da como resultado exactamente 0, significa que ambos usuarios no tienen nada en común.

\begin{equation} \label{eq_cos_sym}
    \cos(\theta) = {\mathbf{A} \cdot \mathbf{B} \frac \|\mathbf{A}\| \|\mathbf{B}\|} = \frac{ \sum\limits_{i=1}^{n}{A_i  B_i} }{ \sqrt{\sum\limits_{i=1}^{n}{A_i^2}}  \sqrt{\sum\limits_{i=1}^{n}{B_i^2}} }   
\end{equation}

\subsubsection{Estilo de Participación}
[?]
% - Esta función está pensada para que pueda haber fluidez para hablar dentro del grupo; minimizando los silencios y aumentando volviendo el número de participaciones uniforme con respecto a los usuarios.
% - Hay mucha literatura mencionando porque esto funciona, pero nadie dice una medida matemática que se pueda usar para esto...
% - Se hizo un análisis de los datos a la base de datos de AMI, y se descubrió que los silencios y las participaciones están directamente relacionados con respecto a una medida de participación dentro del grupo, en efecto, se necesita.
% - TODOS los usuarios se comportan como protagonistas en algún punto, ya que este es el estilo que nos interesa optimizar es en el que nos basamos.
% - Más adelante al momento de generar la base de datos esto también hace sentido y así... 

\section{Algoritmos mono-objetivo y multiobjetivo}
[?]
\subsection{Búsqueda local}
[?]
\subsection{Algoritmo Genético Mono-objetivo}
[?]
\subsection{NSGA-II}
[?]










